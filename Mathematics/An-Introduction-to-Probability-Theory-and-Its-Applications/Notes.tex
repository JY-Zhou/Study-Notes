\documentclass{article}
\author{Jianyu Zhou}
\title{\textbf{Study Notes}\\ \textit{An Introduction of Probability Theory and Its Applications}}
\begin{document}
	\maketitle
	\section{The Sample Space}
	\subsection{The Empirical Background}
	\paragraph{Events} Events are the results of experiments or obervations.
	The events should be distinguished between \textit{compound(decomposable)} and \textit{simple(indecomposable) events}. A compound event is an aggregate of certain simple events.
	\paragraph{Sample points} Sample points are just the simple events. Every indecomposable result of the (idealized) experiment is represented by one, and only one, sample point.
	\paragraph{Sample Space} Sample space is the aggregate of all sample points.
	\subsection{Examples}
	\paragraph{Notes} 
	\begin{enumerate} 
	\item All intuitive background sampling problems are abstractly equivalent to the scheme of \textbf{placing \textit{r} balls into \textit{n} cells}, in the sense that the outcomes differ only in their verbal description. \\There are some examples listed on P10.
	\item Here comes a question that how to calculate the size of sampling space (how many sampling points in the sampling space) when placing \textit{r} balls into \textit{n} cells? This question should be answered under different assumption.
		\begin{itemize}
		\item Placing \textit{r} \textbf{distinguishable} balls into \textit{n} \textbf{distinguishable} cells. \\ $$N=n^r$$
		\item Placing \textit{r} \textbf{indistinguishable} balls into \textit{n} \textbf{distinguishable} cells. \\ $$?$$
		\item Placing \textit{r} \textbf{distinguishable} balls into \textit{n} \textbf{indistinguishable} cells. \\ $$?$$
		\item Placing \textit{r} \textbf{indistinguishable} balls into \textit{n} \textbf{indistinguishable} cells. \\ $$?$$
		\end{itemize}
	When we facing some particular problems, the model of distinguishable or indistinguishable balls is purely a matter of purpose and convenience.
	\item Sample space is the universe of the sample points. All sample points are indecomposable and cover all the sample results (even if the result is impossible). The event is an aggregate of some sample points (one or the aggregate of all sample points is also an event). 
	\end{enumerate}
	\subsection{The Sample Space. Events}
\end{document}