\documentclass{article}
\author{Jianyu Zhou}
\title{\textbf{Study Notes}\\ \textit{An Introduction of Probability Theory and Its Applications}}
\begin{document}
	\maketitle
	\section{The Sample Space}
	\subsection{The Empirical Background}
	\paragraph{Events} Events are the results of experiments or obervations.
	The events should be distinguished between \textit{compound(decomposable)} and \textit{simple(indecomposable) events}. A compound event is an aggregate of certain simple events.
	\paragraph{Sample points} Sample points are just the simple events. Every indecomposable result of the (idealized) experiment is represented by one, and only one, sample point.
	\paragraph{Sample Space} Sample space is the aggregate of all sample points.
	\paragraph{Notes}
	\paragraph{1.} It seems that all the sampling problems can be abstract into placing \textit{r} balls into \textit{n} cells. Read in depth! (P10-11)
\end{document}