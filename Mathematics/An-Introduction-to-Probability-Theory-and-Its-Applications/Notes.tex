\documentclass{article}
\usepackage{amsmath}
\numberwithin{equation}{subsection}
\author{Jianyu Zhou}
\title{\textbf{Study Notes}\\ \textit{An Introduction of Probability Theory and Its Applications}}
\begin{document}
	\maketitle
	\tableofcontents
	\newpage
	\section{The Sample Space}
		\subsection{The Empirical Background}
			\paragraph{Events} Events are the results of experiments or obervations. The events should be distinguished between \textit{compound} (\textit{decomposable}) and \textit{simple} (\textit{indecomposable events}). A compound event is an aggregate of certain simple events.
			\paragraph{Sample points} Sample points are just the simple events. Every indecomposable result of the (idealized) experiment is represented by one, and only one, sample point.
			\paragraph{Sample Space} Sample space is the aggregate of all sample points.
		\subsection{Examples}
			\paragraph{Notes} 
			\begin{enumerate} 
			\item All intuitive background sampling problems are abstractly equivalent to the scheme of \textbf{placing \textit{r} balls into \textit{n} cells}, in the sense that the outcomes differ only in their verbal description. \\(There are some examples listed on P10.)
			\item Here comes a question that how to calculate the size of sampling space (how many sample points in the sample space) when placing \textit{r} balls into \textit{n} cells? This question should be answered under different situations.
				\begin{itemize}
				\item Placing \textit{r} \textbf{distinguishable} balls into \textit{n} \textbf{distinguishable} cells. \\ $$N=n^r$$
				\item Placing \textit{r} \textbf{indistinguishable} balls into \textit{n} \textbf{distinguishable} cells. \\ $$N= {r+n-1 \choose r}$$
				\item Placing \textit{r} \textbf{distinguishable} balls into \textit{n} \textbf{indistinguishable} cells. \\ $$?$$
				\item Placing \textit{r} \textbf{indistinguishable} balls into \textit{n} \textbf{indistinguishable} cells. \\ $$?$$
				\end{itemize}
			When we facing some particular problems, the model of distinguishable or indistinguishable balls is purely a matter of purpose and convenience.
			\end{enumerate}
		\subsection{The Sample Space and Events} 
			\paragraph{} Sample space is the universe of the sample points. All sample points are indecomposable and cover all the outcomes of an idealized experiment. The event is an aggregate of some sample points (one or the aggregate of all sample points is also an event).
			\paragraph{} New events can be define in terms of two or more given events. Here comes the notation of the formal \textit{algebra of events} (algebra of point sets).
		\subsection{Relations among Events}
			\paragraph{} We use $\Omega$ to denote sample space and capitals to denote events (sets of sample points). The fact that a sample point $x$ is contained in the event $A$ is denoted by $x \in A$. Thus $x \in \Omega$ for every sample point $x$. We write $A=B$ only if the two events consist of exactly the same points.
			\paragraph{Definations} 
				\begin{enumerate}
				\item We shall use the notation $A=0$ to express that the event $A$ contains no sample points. The zero must be interpreted in \textbf{a symbolic sense and not as the numeral}.
				\item The event consisting of all points not contained in the event $A$ will be called the \textit{complementary event }(or \textit{negation}) and will be denoted by $A'$. In particular $\Omega ' = 0$.
				\item With any two events $A$ and $B$ we can associate two new events defined by the conditions ``\textit{both $A$ and $B$ occur}'' and ``\textit{either $A$ or $B$ or both occur}''. These events will be denoted by $AB$ (intersection of $A$ and $B$) and $A\cup B$ (union of $A$ and $B$). If $A$ and $B$ exclude each other, then there are no points common to $A$ and $B$ and the event $AB$ is impossible; analytically this situation is described by the equation $AB=0$ which should be read ``$A$ and $B$ are \textit{mutually exclusive}''.
				\item To every collection $A,B,C,\dots$ of events we define some notions as  follows.
					\begin{itemize}
					\item The aggregate of sample points which belong to all the given sets will be denoted by $ABC\dots$ and called the \textit{intersection} (or \textit{simultaneous realization}).
					\item The aggregate of sample points which belong to at least one of the given sets will be denoted by $A\cup B\cup C\cup \dots$ and called the \textit{union} (or \textit{realization of at least one}).
					\item The events $A,B,C,\dots$ are \textit{mutually exclusive} if no two have a point in common (any two of the events are mutually exclusive), that is, if $AB=0, AC=0, \dots, BC=0, \dots$.
					\end{itemize}
				\item The symbols $A\subset B$ and $B\supset A$ signify that every point of $A$ \textit{is contained in} $B$; they are read, respectively, \textit{$A$ implies $B$} and \textit{$B$ is implied by $A$}. \textbf{If this is the case}, we shall also write $B-A$ instead of $BA'$ to denote the event that \textit{$B$ but not $A$ occurs}. The event $B-A$ contains all those points which are in $B$ but not in $A$. With this notation we can write $A'=\Omega-A$ and $A-A=0$.
					\begin{itemize}
					\item If $A$ and $B$ are mutually exclusive, then the occurrence of $A$ implies the non-occurrence of $B$ and vice versa. Thus $AB=0$ means the same as $A\subset B'$ and as $B\subset A'$.
					\item The event $A-AB$ means the occurrence of $A$ but not of the $A$ and $B$. Thus $A-AB=AB'$.
					\end{itemize}
				\end{enumerate} 
		\subsection{Discrete Sample Spaces}
			\paragraph{Discrete Sample Space} A sample space is called discrete if it contains only finitely many points or infinitely many points which can be arranged into a simple sequence $E_1,E_2\dots$. 
			\paragraph{} The probabilities of events in discrete sample spaces are obtained by mere additions, whereas in other spaces integrations are necessary.
		\subsection{Probabilities in Discrete Sample Spaces: Preparations}
			\paragraph{} The notion of probability is based on an \textit{idealized model}. In a sense, no ideal models exist in reality. But in many applications, it is sufficiently accurate to describe reality and the idealized model can be extremely useful when the experiment is impossible to reconstruct.
			\paragraph{} Here comes some interesting examples. When placing 3 distinguishable balls into 3 cells, there will be 27 distinguishable possibilities. It appears natural to assume that all sample points are \textit{equally probable}, that is, that each sample point has probability $\frac{1}{27}$. If the 3 balls are indistinguishable, there remain 27 different possibilities, even though only ten different forms are distinguishable. This consideration leads us to attribute the following probabilities to this ten points as following Table \ref{1.6:Table1}.
			\begin{table}
				\begin{center}
					\begin{tabular} {| c | c | c | c |}
						\hline
						Result & Probability & Result & Probability \\ \hline
						$\{***|\:-\:|\:-\:\}$ & $\frac{1}{27}$ & $\{\:*\:|*\:*|\:-\:\}$ & $\frac{1}{9}$ \\ \hline
						$\{\:-\:|***|\:-\:\}$ & $\frac{1}{27}$ & $\{\:*\:|\:-\:|*\:*\}$ & $\frac{1}{9}$ \\ \hline
						$\{\:-\:|\:-\:|***\}$ & $\frac{1}{27}$ & $\{\:-\:|*\:*\:|\:*\:\}$ & $\frac{1}{9}$ \\ \hline
						$\{*\:*|\:*\:|\:-\:\}$ & $\frac{1}{9}$ & $\{\:-\:|\:*\:|*\:*\}$ & $\frac{1}{9}$ \\ \hline
						$\{*\:*|\:-\:|\:*\:\}$ & $\frac{1}{9}$ & $\{\:*\:|\:*\:|\:*\:\}$ & $\frac{2}{9}$\\ \hline			
					\end{tabular}
					\caption{The probability distribution}
					\label{1.6:Table1}
				\end{center}
			\end{table}
			However, in our case with $r=n=3$, Bose and Einstein showed that certain particles attribute equal probability $\frac{1}{10}$ to each of the ten sample points.
			\paragraph{} These examples teach us not to rely too much on a priori arguments and to be prepared to accept new and unforeseen schemes.
		\subsection{The Basic Definitions and Rules}
			\paragraph{Fundamental Convention} Given a discrete sample space $\Omega$ with sample points $E_1, E_2, \dots$, we shall assume that with each point $E_j$ there is associated a number, called the \textit{probability} of $E_j$ and denoted by $P\{E_j\}$. It is to be non-negative and such that
			\begin{equation}
				\label{eq:1.7.1}
				P\{E_1\}+P\{E_2\}+\dots=1.
			\end{equation}
			\paragraph{}Note that the point which probability is zero (which is impossible to occur) is not excluded. In discrete sample spaces, any sample point can be eliminated from the sample space. However, the numerical values of probabilities are not known in advance.
			\paragraph{Definition} The probability $P\{A\}$ of any event $A$ is the sum of the probabilities of all sample point in it.
			\paragraph{} For the sample space, 
			\begin{equation}
				\label{eq:1.7.2}
				P\{\Omega\}=1.
			\end{equation}
			It follows that for any event $A$, 
			\begin{equation}
				\label{eq:1.7.3}
				0\leq P\{A\} \leq 1.
			\end{equation}
			Consider now two arbitrary events $A_1$ and $A_2$. To compute the probability $P\{A_1\cup A_2\}$ that either $A_1$ or $A_2$ or both occur, we have to add the probabilities of all sample points contained either in $A_1$ or in $A_2$, but each point is to be counted only once. Therefore, 
			\begin{equation}
				\label{eq:1.7.4}
				P\{A_1\cup A_2\} \leq P\{A_1\}+P\{A_2\}.
			\end{equation}
			\paragraph{Theorem} For any two events $A_1$ and $A_2$ the probability that either $A_1$ or $A_2$ or both occur is given by 
			\begin{equation}
				\label{eq:1.7.5}
				P\{A_1\cup A_2\} = P\{A_1\}+P\{A_2\}-P\{A_1A_2\}. 
			\end{equation}
			If $A_1A_2 = 0$, that is, if $A_1$ and $A_2$ are mutually exclusive, then the formula reduces to 			
			\begin{equation}
				\label{eq:1.7.6}
				P\{A_1\cup A_2\} = P\{A_1\}+P\{A_2\}.
			\end{equation}			
			The probability $P\{A_1\cup A_2\cup \dots \cup A_n\}$ of the realization of at least one among $n$ events can be computed by a formula analogous to the Formula \ref{eq:1.7.5} and will be discussed later. For arbitrary events $A_1,A_2,\dots$ the inequality
			\begin{equation}
				\label{eq:1.7.7}
				P\{A_1\cup A_2\cup \dots\} \leq P\{A_1\} + P\{A_2\} + \dots
			\end{equation}
			holds (Occasionally \ref{eq:1.7.7} referred to as \textit{Boole's inequality}). In the special case where the events $A_1,A_2,\dots$ are mutually exclusive, we have 
			\begin{equation}
				\label{eq:1.7.8}
				P\{A_1\cup A_2\cup \dots\} = P\{A_1\} + P\{A_2\} + \dots
			\end{equation}
			\paragraph{} Finally, the probabilities of all sample points in sample spaces should not be same. Although considering them sharing the same probability will simplify the problem, this assumption will be restricted almost entirely the study of games.
		\subsection{Problems Solution}
			\begin{enumerate}
			\item (a) $P\{A\} = \frac{3}{5}$, \\ (b) $P\{B\} = \frac{3}{5}\times\frac{2}{4} + \frac{2}{5}\times\frac{3}{4} = \frac{3}{5}$ \\ (c) $P\{C\}=\frac{3\times2}{20} = \frac{3}{10}$  
			\item 6. With the number of points of $S_1, S_2, S_1\cup S_2$ are $12, 12, 18$ respectively, the number of points of $S_1S_2 = 12 + 12 - 18 = 6$. 
			\item $P\{A_1\} = P\{A_2\} = P\{A_3\} = P\{A_4\} = \frac{3\times2\times1}{24} = \frac{1}{4}$, \\ $P\{A_1\cup A_2\}=\frac{3\times2\times1}{24} + \frac{2\times2\times1}{24} = \frac{5}{12}$, (the aggregate of the points which the first place is 1 and the points which the second place is 2 but the first place is not 1) \\ $P\{A_1A_2\} = \frac{2\times1}{24} = \frac{1}{12}$, \\
			so, $P\{A_1\cup A_2\} = P\{A_1\} + P\{A_2\} - P\{A_1A_2\}$ is valid. \\ \dots 
			\item The sample space is all of the outcomes tossing the coin for $N$ times, which contains $2^N$ sample points. Here $N$ is an infinite number. So, let $E_n$ denote the experiment ends after $n (n\geq2)$ tossing, and 
			\begin{equation}
				P\{E_n\} = \frac{2 \times 2^{N-n}}{2^n \times 2^{N-n}} = \frac{1}{2^{n-1}}.
			\end{equation}
			Note that $E_2, E_3,\dots,E_n,\dots$ are mutually exclusive. \\(a) So 
			\begin{equation}
				\begin{aligned}
					P\{A\} &= P\{E_2 \cup E_3 \cup E_4 \cup E_5\} \\
					 &= P\{E_2\} + P\{E_3\} + P\{E_4\} + P\{E_5\} \\ 
					 &= \frac{1}{2} + \frac{1}{4} + \frac{1}{8} + \frac{1}{16} = \frac{15}{16}.
				\end{aligned}
			\end{equation}
			\\(b) Similarly, with $k$ is an positive integer,
			\begin{equation}
				\begin{aligned}
					P\{B\} &= P\{E_2 \cup E_4 \cup \dots \cup E_{2k} \cup \dots\} \\
					&= P\{E_2\} + P\{E_4\} + \dots + P\{E_{2k}\} + \dots \\
					&= \frac{1}{2} + \frac{1}{8} + \dots + \frac{1}{2^{2k-1}} + \dots \\
					&=\sum_{k=1}^{\infty}\frac{1}{2^{2k-1}} = \lim_{k \to \infty}\frac{\frac{1}{2}\left(1-\frac{1}{4^k}\right)}{1-\frac{1}{4}} = \frac{2}{3}
				\end{aligned}
			\end{equation}
			\\ \textbf{Notes:} This solution is not totally correct. Although the answers are correct, the sample space is not corresponding to the description. The sample space should contain the two points $HH$ and $TT$ with probability $\frac{1}{4}$; the two points $HTT$ and $THH$ with probability $\frac{1}{8}$; and generally two point with probability $\frac{1}{2^n}$ with $n\geq2$. So, in conclusion, the sample points are the \textbf{outcomes} of the experiment and each point will have it \textbf{own's} probability.
			\item (a) $P\{Unity\} = 2\times(\frac{1}{4} + \frac{1}{8} + \dots + \frac{1}{2^k} + \dots) = 1.$ (Under the description the unity the points (*) is just the sample space.) \\ (b) Let $A,B,C$ denote the events that a, b or c wins respectively. Note that if nobody win the game, the loop shall be either $acbacb\dots$ or $bcabca\dots$. So, a, b winning the game is the either one of the points when there are $3n-1$ or $3n+1$ characters $(n = 1,2,3\dots)$, i.e.
			\begin{equation}
				\begin{aligned}
					P\{A\} = P\{B\} &= \frac{1}{4} + \frac{1}{16} + \dots + \frac{1}{2^{3n-1}} + \frac{1}{2^{3n+1}} + \dots \\ &= \sum_{n=1}^{\infty}\left(\frac{1}{2^{3n-1}}+\frac{1}{2^{3n+1}}\right)\\ &= \lim_{n \to \infty}\frac{\frac{1}{4}\left(1-\frac{1}{8^n}\right)}{1-\frac{1}{8}}+\lim_{n \to \infty}\frac{\frac{1}{16}\left(1-\frac{1}{8^n}\right)}{1-\frac{1}{8}}\\ &= \frac{5}{14}
				\end{aligned}
			\end{equation}
			and c winning the game is the both points when there are $3n$ characters $(n=1,2,3\dots)$, i.e.
			\begin{equation}
				\begin{aligned}
					P\{C\} &= 2 \times \left(\frac{1}{8} + \frac{1}{64} + \dots + \frac{1}{2^{3n}} + \dots \right)\\ &= \sum_{n=1}^{\infty}{\frac{2}{2^{3n}}} \\ &= \lim_{n \to \infty}\frac{\frac{2}{8}\left(1-\frac{1}{8^n}\right)}{1-\frac{1}{8}} = \frac{2}{7}
				\end{aligned}
			\end{equation}
			\\ (c) If no decision is reached at or before the $k$th turn, the game must be end after the $k$th turn. So we can consider the complementary event, i.e.
			\begin{equation}
				\begin{aligned}
					P\{D\} &= 1-P\{D'\}\\ &=1- 2 \times \left(\frac{1}{2^{k+1}} + \frac{1}{2^{k+2}} + \dots \right)\\ &= 1- \sum_{n=k}^{\infty}{\frac{1}{2^{n}}} \\ &= \lim_{n \to \infty}\frac{\frac{1}{2^k}\left(1-\frac{1}{2^n}\right)}{1-\frac{1}{2}} = \frac{1}{2^{k-1}}
				\end{aligned}
			\end{equation}
			\textbf{Note: } This is a good problem, and please review this problem in detail. It tells me that before we calculate the probability of the event, we should \textbf{determine what the sample space is and the probability distribution of the sample points are in advance}. Then we should \textbf{analyze the specific sample points which are included in the event strictly}. Finally, \textbf{take more care about the complementary events and whether the events are mutually exclusive}. This is the correct way to calculate the probability of the event.
			\item \textbf{?}
			\item It is obvious that $A_1A_2A_3 = A_1A_2A_3A_4 \subset A_4$ and $A_1A_2A_3' = A_1A_2A_3'A_4' \subset A_4'$.
			\item (a) It is obvious that $S_1S_2$ and $D_3$ are mutually exclusive, so $S_1S_2D_3 = 0$. \\ (b) Because there are 3 balls, $S_1D_2 = S_1D_2E_3 \subset E_3$ holds. \\ (c) There are only 3 balls, so $E_3 = S_1D_2 \cup S_2D_1$ and $(S_1D_2)(S_2D_1) = 0$. Then $E_3-D_2S_1 \supset S_2D_1$ holds.
			\item (a) $AB$ contains the points $(1,2), (1,4), (1,6), (2,1), (4,1), (6,1)$, so $P\{AB\} = \frac{1}{6}$. \\ (b) $P\{A \cup B\} = P\{A\} + P\{B\} - P\{AB\} = \frac{1}{2} + \frac{11}{36} - \frac{1}{6} = \frac{23}{36}$. \\ (c) $P\{AB'\} = \frac{12}{36} = \frac{1}{3}$.
			\item (a) The husband and wife are both older than 40 and the husband is older than wife. \\ (b) The husband is older than 40 but younger than wife. \\ (c) The husband and wife are both older than 40 and the husband is younger than wife.
			\item $AC'$ indicates that the husband is older than 40 but the wife is younger than 40, so the husband must be older than wife and $B$ holds.
			\item (a) 0. (b) 0. (c) 4. (d) 2. (e) 1. (f) 1. (g) 0.
			\item The verifications are easy to think up considering the description of the problem.
			\item The solution can be got by drawing some figures easily. Here I copy these formulas to enhance the impression. 
			\\ (a) $(A\cup B)' = A'B'$ \\ (b) $(A\cup B)-B = A-AB = AB'$ \\ (c) $AA = A\cup A = A$ \\ (d) $(A-AB)\cup B = A \cup B$ \\ (e) $(A\cup B) - AB = AB' \cup A'B$ \\ (f) $A'\cup B' = (AB)'$ \\ (g) $(A\cup B)C = AC\cup BC$
			\item (a) $(A\cup B)(A\cup B') =(A\cup B)A \cup (A\cup B)B'=AA\cup BA\cup AB' \cup BB' = A \cup AB' = A$ \\ (b) $(A\cup B)(A'\cup B)(A\cup B') = B(A\cup B') = BA \cup BB' = AB$ \\ (c) $(A\cup B)(B\cup C) = AB \cup BB\cup AC\cup BC = AB\cup AC\cup BC$
			\item True:  \\False: 
			\end{enumerate}
	\newpage
	\section{Elements of Combinatorial Analysis}
		\subsection{Preliminary}
			todo\dots
			
		
\end{document}
